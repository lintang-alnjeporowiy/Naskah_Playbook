\documentclass[12pt]{article}
\usepackage[a4paper,margin=1in]{geometry}
\usepackage{enumitem}
\usepackage{titlesec}
\usepackage{lmodern}
\usepackage{setspace}
\usepackage{hyperref}
\usepackage{amssymb}

\titleformat{\section}{\normalfont\Large\bfseries}{\thesection.}{1em}{}
\titleformat{\subsection}{\normalfont\large\bfseries}{\thesubsection.}{1em}{}

\setlength{\parindent}{1.5em}
\setlength{\parskip}{0.5em}
\onehalfspacing
\title{\textbf{Playbook iStow Writing Style Guide} \\[0.5em] 
       \large As summarized from the book ``Naval Architecture for Non-Naval Architects''}
\author{Lintang Fitri}
\date{7 Mei 2025}

\begin{document}

\maketitle

\section{Key Features of the Style}

\subsection{Conversational Opening}
Begins with a rhetorical question or personal appeal.

\textbf{Example:} \textit{"Do you have the bad habit of skipping the preface in books?"}

\subsection{Clear Sectioning and Headings}
\begin{itemize}
  \item Uses numbered sections with italicized or bolded titles.
  \item Organized in logical progression: definition → classification → requirements → terminology.
\end{itemize}

\subsection{Direct Address to the Reader}
Uses second-person pronouns like \textit{you} to create a sense of dialogue.

\subsection{Simplified Language for Clarity}
\begin{itemize}
  \item Avoids jargon or explains it immediately.
  \item \textbf{Example:} \textit{"Generally speaking, a ship is big, a boat is small."}
\end{itemize}

\subsection{Examples and Analogies}
Connects technical ideas with relatable images.

\textbf{Example:} \textit{"A deck is like a floor of a building."}

\subsection{Use of Lists and Enumeration}
\begin{itemize}
  \item Uses bullet points or alphabetic enumeration.
  \item Helps break complex information into digestible parts.
\end{itemize}

\subsection{Brief Definitions}
Technical terms are introduced concisely.

\textbf{Example:} \textit{"Bow: the front end."}

\subsection{Historical or Fun Facts}
Adds interest and variety.

\textbf{Example:} \textit{"The origin of the long ton comes from wine barrels."}

\section{ How to Write in This Style}

\subsection{Start Engagingly}
Begin with a question, anecdote, or casual remark.

\textbf{Example:} \textit{"Ever wondered why ships don’t tip over?"}

\subsection{Structure Clearly}
Use familiar section headers such as:
\begin{itemize}
  \item Introduction
  \item Basic Principles
  \item Key Definitions
  \item Summary or Applications
\end{itemize}

\subsection{Write as if Explaining to a Curious Beginner}
\begin{itemize}
  \item Avoid assuming prior knowledge.
  \item Define every technical term at first use.
\end{itemize}

\subsection{Use Examples Often}
Relate abstract ideas to everyday life.

\textbf{Example:} \textit{"Think of a hull as the skin of the ship, keeping water out like a raincoat."}

\subsection{Keep Sentences Short and Active}
Prefer clarity over complexity.

\textbf{Example:} \textit{"The hull must be strong but not too heavy."}

\subsection{Sprinkle in Personality}
Use light humor or human touches.

\textbf{Example:} \textit{"Don’t worry, this won’t be on a test (probably)."}

\section{Structural Approach}
Since the main goal is to create a book that isn't boring, although I doubt the TikTok generation can stand more than 10 minutes in front of a book, the content structure will also be more flexible, with many boxes/inserts containing additional information related to what's discussed in a chapter. Infographics will also be added where possible, though looking at our current manpower, this seems like it will be quite the challenge.


\end{document}
